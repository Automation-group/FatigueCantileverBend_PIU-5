% первая глава
\section{О программе CalcDeflection}

\smallskip %\medskip \bigskip
Программа CalcDeflection предназначена для расчёта величины прогиба
образца при заданном механическом напряжении $\sigma$. Результат расчёта
величины прогиба при заданном $\sigma$ используется при испытаниях материаллов
на усталось в установках ПИУ-4 и ПИУ-5. В программе можно задать размеры образца,
выбрать ось по которой будет приложена сила изгибающая образец и форуму
сечения рычага к которому прикладывается изгибащая сила. Материалы из
которых изготовлены образец и рычаг задаются установкой модуля Юнга.
Кроме этого в программе можно вывести время расчёта, шаг интегрирования и
некоторые другие параметры о которых будет рассказано в следующей главе.

\ Поскольку для написания программы была использован кроссплатформенный фреймворк $Qt$,
то программа может быть собрана под различные операционные системы. Внешнишний вид
программы для \mbox{ОС Kubutu 22.04} представлен на Рис.\ref{fig:calcDeflection_linux}, а для
\mbox{ОС Windows 10} на Рис.\ref{fig:calcDeflection_windows}.

\begin{figure}[h]
	\begin{center}
		\hspace*{\fill}
		\begin{minipage}[ht]{0.4\linewidth}
			\includegraphics[width=1.0\textwidth]{image/CalcDeflection_screen_linux.png}
			\caption{\small Внешний вид CalcDeflection v0.2 в Kubuntu 24.04}
			\label{fig:calcDeflection_linux}
		\end{minipage}
		\hfill
		\begin{minipage}[ht]{0.4\linewidth}
			\includegraphics[width=1.0\textwidth]{image/CalcDeflection_screen_windows.png}
			\caption{\small Внешний вид CalcDeflection v0.2 в Windows 10}
			\label{fig:calcDeflection_windows}
		\end{minipage}
		\hspace*{\fill}
	\end{center}
\end{figure}

\ Исходный код программы можно взять на сайте
\href{https://github.com/Automation-group/FatigueCantileverBend_PIU-5/tree/main/programs/CalcDeflection}{github.com‎}.
Описание по сборке программы из исходников находится по этой же ссылке.
