% первая глава
\section{Алгоритм расчёта величины прогиба}

\subsection{Дифференциальное уравнение для функции прогиба}

Для расчёта величины прогиба воспользуемся системой дифференциальных
уравнений из \cite{1_} $\S$  8.1 и  $\S$  8.2. Система дифференциальных уравнений первого порядка для
функции прогиба имеет вид

\begin{align}
	\label{eq:sysEquat_deflection_full}
  	\begin{cases}
		\dfrac{d \varphi}{dx} &= - \dfrac{M}{E \cdot J} \\[3ex]
		\dfrac{d \nu}{dz} &= \tg \, \varphi
  	\end{cases}
\end{align}

\begin{flushleft}
	$\varphi$ - угол прогиба в радианах,\newline
	$\nu$ - величина прогиба в метрах,\newline
	$M$ - изгибающий момент в Нм,\newline
	$E$ - модуль Юнга в Па,\newline
	$J$ - момент инерции сечения в кг$\cdot$м$^2$\newline
\end{flushleft}

Посколюку угол $\varphi$ выражен в радианах, то при малых углах прогиба,
менее 0.5 радиан, \mbox{$\tg \, \varphi \approx \varphi$} и поэтому имеем простую
зависиммость между функцией углов поворота сечений и функцией прогибов

\begin{align}
	\label{eq:sysEquat_deflection}
  	\begin{cases}
		\dfrac{d \varphi}{dx} &= - \dfrac{M}{E \cdot J} \\[3ex]
		\dfrac{d \nu}{dz} &= \varphi
  	\end{cases}
\end{align}

Из ситемы уравнений \eqref{eq:sysEquat_deflection} выразим величину прогиба $\nu$

\begin{align}
	\label{eq:iint_deflection}
  	\nu &=  \int\limits_{0}^{L} \int\limits_{0}^{z'}  \frac{M(z)}{E(z) \cdot J(z)} \,dz \,dz
\end{align}

По формуле (\ref{eq:iint_deflection}) рассчитывается значение модуля прогиба,
то есть знак "$-$" \ опущен. Переменная $L$ это суммарную длину
рабочей части образца и рычага в метрах \eqref{eq:L_Lp_Ls}. Кроме этого
в подинтегральном выражении значения изгибающего момента $M(z)$,
момента инерции сечения $J(z)$ и величина модуля Юнга $E(z)$ зависят от
координаты $z$, а $J(z)$ ещё зависит от формы образца с рычагом и
направления приложения силы. То есть возникает несколько вариантов
для вычисления величины прогиба $\nu$ в зависимости от формы сечения рычага -
круглой или прямоугольной и направления приложения силы - по оси $x$
или по оси $y$. Рассмотрим эти варианты подробнее.

Сначала рассмотрим образец прикреплённый к рычагу прямоугольного сечения.
На Рис.\ref{fig:rectangularLeash} представлены два чертежа с приложенной
к рычагу силой $\vec{P}$ по \mbox{оси $x$} и по \mbox{оси $y$}.

\newpage
\begin{figure}[h]
	\begin{center}
		\hspace*{\fill}
		\begin{subfigure}{0.4\textwidth}
			\includesvg[width=\textwidth]{image/sampleRadiusWorkingPart_rectangularLeash_force_X-axis.svg}
			\caption {Сила приложена по оси $x$}
			\label{fig:rectangularLeash_force_X-axis}
		\end{subfigure}
		\hfill
		\begin{subfigure}{0.4\textwidth}
			\includesvg[width=\textwidth]{image/sampleRadiusWorkingPart_rectangularLeash_force_Y-axis.svg}
			\caption {Сила приложена по оси $y$}
			\label{fig:rectangularLeash_force_Y-axis}
		\end{subfigure}
		\hspace*{\fill}
		\caption {Образец с поводком прямоугольного сечения и силой приложенной по оси $y$}
		\label{fig:rectangularLeash}
	\end{center}
\end{figure}

Распишем подробнее переменные $M(z)$, $E(z)$ и $J(z)$ из подинтегральное выражение
формулы \eqref{eq:iint_deflection} для образца прикреплённого к рычагу
прямоугольного сечения с приложением силы по оси $x$, как показанано на
Рис. \ref{fig:rectangularLeash_force_X-axis}.

\begin{align}
	\label{eq:Mz_rect_x-axis}
	&M(z) = P(L-z) \\
	\label{eq:Ez_rect_x-axis}
	&E(z) =
	\begin{cases}
		\text{Модуль Юнга материала образца,} &\text{если }\ 0 \leqslant z < L \_ s \\
		\text{Модуль Юнга материала рычага,} &\text{если }\ L \_ s \leqslant z \leqslant L
	\end{cases}
	\\
	\label{eq:Jz_rect_x-axis}
	&J(z) =
	\begin{cases}
		\dfrac{2}{3} \cdot b \_ s^3 \cdot y(z) \text{,} &\text{если }\ 0 \leqslant z < L \_ s \\[3ex]
		\dfrac{b \_ p^3 \cdot a \_ p}{12} \text{,} &\text{если }\ L \_ s \leqslant z \leqslant L
	\end{cases}
\end{align}

Величина $L$ это суммарная длина рабочей части образца и рычага

\begin{align}
	\label{eq:L_Lp_Ls}
	L = L \_s + L \_ p
\end{align}

Зависимость $y(z)$ определяется геометрией образца и рассчитывается из системы уравнений

\begin{align}
	\label{eq:yz_R_s_y_0_z_0}
	\begin{cases}
		y(z) &= y_0 - \sqrt{R_s^2-(z - z_0)^2} \\[2ex]
		y_0 &= R_s + \dfrac{h \_ s}{2} \\[2ex]
		z_0 &= \dfrac{L \_ s}{2} \\[2ex]
		R_s &= \dfrac{L \_ s^2 +(H \_ s - h \_ s)^2}{4 \cdot (H \_ s - h \_ s)}
	\end{cases}
\end{align}

В системе уравнений \eqref{eq:yz_R_s_y_0_z_0} $y_0$ и $z_0$ это координаты
центра радиуса скругления образца, а $R_s$ это радиус скругления. Изображение образца
с указанием начала координат и центром скругления показано на Рис. .

%\begin{wrapfigure}{2}{200pt}
%    \includesvg[width=2.3in]{image/sampleRadiusWorkingPart_rectangularLeash_force_X-axis.svg}
%    \caption {Образец с поводком прямоугольного сечения и силой приложенной по оси $x$}
%    \label{fig:rectangularLeash_force_X-axis}
%\end{wrapfigure}
