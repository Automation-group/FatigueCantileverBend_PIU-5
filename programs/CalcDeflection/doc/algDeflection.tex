% первая глава
\section{Алгоритм расчёта величины прогиба}

\subsection{Дифференциальное уравнение для функции прогиба}

Для расчёта величины прогиба воспользуемся системой дифференциальных
уравнений из  $\S$  8.2 \cite{1_}. Система дифференциальных уравнений первого порядка для
функции прогиба имеет вид

\begin{align}
	\label{eq:sysEquat_deflection}
  	\begin{cases}
		\dfrac{d \phi}{dx} &= - \dfrac{M}{E \cdot J} \\[3ex]
		\dfrac{dW}{dz} &= \phi
  	\end{cases}
\end{align}

\begin{wrapfigure}{2}{200pt}
    \includesvg[width=2.3in]{image/sampleRadiusWorkingPart_rectangularLeash_force_X-axis.svg}
    \caption {Образец с поводком прямоугольного сечения и силой приложенной по оси $x$}
    \label{fig:rectangularLeash_force_X-axis}
\end{wrapfigure}

\begin{flushleft}
	$\phi$ - угол прогиба в градусах,\newline
	$W$ - величина прогиба в метрах,\newline
	$M$ - изгибающий момент в Нм,\newline
	$E$ - модуль Юнга в Па,\newline
	$J$ - момент инерции сечения в кг$\cdot$м$^2$\newline
\end{flushleft}

Из ситемы уравнений (\ref{eq:sysEquat_deflection}) выразим величину прогиба $W$

\begin{align}
	\label{eq:iint_deflection}
  	W &=  \int\limits_{0}^{L} \int\limits_{0}^{z'}  \frac{M(z)}{E(z) \cdot J(z)} \,dz \,dz
\end{align}

В формуле (\ref{eq:iint_deflection}) рассчитывается значение модуля прогиба,
то есть знак "$-$" \ опущен. Кроме этого в подинтегральном выражении
значения изгибающего момента $M(z)$,
момента инерции сечения $J(z)$ и величина модуля Юнга $E(z)$ зависят от
координаты $z$, а $J(z)$ ещё зависит от формы образца с рычагом и
направления приложения силы. То есть возникает четыре основных варианта
для вычисления величины прогиба $W$ - в зависимости от формы сечения рычага
круглой или прямоугольной и направления приложения силы - по оси $x$
или по оси $y$. Рассмотрим эти варианты подробнее.

Сначала рассмотрим вариант представленный на Рис.\ref{fig:rectangularLeash_force_X-axis}
с рычагом прямоугольного сечения и силой приложенной по оси $x$.
% \newpage

