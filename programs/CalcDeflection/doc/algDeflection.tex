% первая глава
\section{Алгоритм расчёта величины прогиба}

\subsection{Дифференциальное уравнение для функции прогиба}

Для расчёта величины прогиба воспользуемся системой дифференциальных
уравнений из \cite{1_} $\S$  8.1 и  $\S$  8.2. Система дифференциальных уравнений первого порядка для
функции прогиба имеет вид

\begin{align}
	\label{eq:sysEquat_deflection_full}
  	\begin{cases}
		\dfrac{d \varphi}{dz} &= - \dfrac{M}{E \cdot J} \\[3ex]
		\dfrac{d \nu}{dz} &= \tg \, \varphi
  	\end{cases}
\end{align}

\begin{flushleft}
	$\varphi$ - угол прогиба в радианах,\newline
	$\nu$ - величина прогиба в метрах,\newline
	$M$ - изгибающий момент в Нм,\newline
	$E$ - модуль Юнга в Па,\newline
	$J$ - момент инерции сечения в кг$\cdot$м$^2$\newline
\end{flushleft}

Поскольку угол $\varphi$ выражен в радианах, то при малых углах прогиба,
менее 0.5 радиан (28.65$\degree$), \mbox{$\tg \, \varphi \approx \varphi$} и поэтому имеем простую
зависиммость между функцией углов поворота сечений и функцией прогибов

\begin{align}
	\label{eq:sysEquat_deflection}
  	\begin{cases}
		\dfrac{d \varphi}{dz} &= - \dfrac{M}{E \cdot J} \\[3ex]
		\dfrac{d \nu}{dz} &= \varphi
  	\end{cases}
\end{align}

Из сиcтемы уравнений \eqref{eq:sysEquat_deflection} выразим величину прогиба $\nu$

\begin{align}
	\label{eq:iint_deflection}
  	\nu &=  \left| \int\limits_{0}^{L} \int\limits_{0}^{z'}  \frac{M(z)}{E(z) \cdot J(z)} \,dz \,dz \right|
\end{align}

По формуле (\ref{eq:iint_deflection}) рассчитывается значение модуля прогиба.
Переменная $L$ это суммарная длина рабочей части образца и рычага в метрах \eqref{eq:L_Lp_Ls},
а $z'$ это та же длина $L$, но по которой в процессе расчёта определяется значение
максимального механического напряжения $\sigma_{max}$ в рабочей части образце.
Кроме этого в подинтегральном выражении значения изгибающего момента $M(z)$,
момента инерции сечения $J(z)$ и величина модуля Юнга $E(z)$ зависят от
координаты $z$, а $J(z)$ ещё зависит от формы образца с рычагом и
направления приложения силы. То есть возникает несколько вариантов
для вычисления величины прогиба $\nu$ в зависимости от формы сечения рычага -
круглой или прямоугольной и направления приложения силы - по оси $x$
или по оси $y$.

\newpage
\subsection{Определение $M(z)$, $E(z)$ и $J(z)$ для образца и рычага}

Рассмотрим образец прикреплённый к рычагу прямоугольного сечения.
На Рис.\ref{fig:rectangularLeash} представлены два чертежа с приложенной
к рычагу силой $\vec{P}$ по \mbox{оси $x$} и по \mbox{оси $y$}.

\begin{figure}[h]
	\begin{center}
		\hspace*{\fill}
		\begin{subfigure}{0.4\textwidth}
			\includesvg[width=\textwidth]{image/sampleRadiusWorkingPart_rectangularLeash_force_X-axis.svg}
			\caption {Сила приложена по оси $x$}
			\label{fig:rectangularLeash_force_X-axis}
		\end{subfigure}
		\hfill
		\begin{subfigure}{0.4\textwidth}
			\includesvg[width=\textwidth]{image/sampleRadiusWorkingPart_rectangularLeash_force_Y-axis.svg}
			\caption {Сила приложена по оси $y$}
			\label{fig:rectangularLeash_force_Y-axis}
		\end{subfigure}
		\hspace*{\fill}
		\caption {Образец с рычагом прямоугольного сечения}
		\label{fig:rectangularLeash}
	\end{center}
\end{figure}

Для расчёта момента инерции сечения $J(z)$ воспользуемся формулами для момента
инерции прямоугольника и круга из \cite{2_} $\S$ 57 и $\S$ 58.
Распишем подробнее переменные $M(z)$, $E(z)$ и $J(z)$ из подинтегрального выражения
формулы \eqref{eq:iint_deflection} для образца прикреплённого к рычагу
прямоугольного сечения с приложением силы по оси $x$, как показанано на
Рис. \ref{fig:rectangularLeash_force_X-axis}.

\begin{align}
	\label{eq:Mz_rect_x-axis}
	&M(z) = P \cdot (L-z) \\
	\label{eq:Ez_rect_x-axis}
	&E(z) =
	\begin{cases}
		\text{Модуль Юнга материала образца,} &\text{если }\ 0 \leqslant z < L_s \\
		\text{Модуль Юнга материала рычага,} &\text{если }\ L_s \leqslant z \leqslant L
	\end{cases}
	\\
	\label{eq:Jz_rect_x-axis}
	&J(z) =
	\begin{cases}
		\dfrac{2}{3} \cdot b_s^3 \cdot y(z) \text{,} &\text{если }\ 0 \leqslant z < L_s \\[3ex]
		\dfrac{b_p^3 \cdot a_p}{12} \text{,} &\text{если }\ L_s \leqslant z \leqslant L
	\end{cases}
\end{align}

Величина $L$ это сумма длин рабочей части образца и рычага до точки приложения
к нему силы $\vec P$

\begin{align}
	\label{eq:L_Lp_Ls}
	L = L_s + L_p
\end{align}

Зависимость $y(z)$ определяется геометрией образца и рассчитывается из системы уравнений

\begin{align}
	\label{eq:yz_R_s_y_0_z_0}
	\begin{cases}
		y(z) &= y_0 - \sqrt{R_s^2-(z - z_0)^2} \\[2ex]
		y_0 &= R_s + \dfrac{h_s}{2} \\[2ex]
		z_0 &= \dfrac{L_s}{2} \\[2ex]
		R_s &= \dfrac{L_s^2 +(H_s - h_s)^2}{4 \cdot (H_s - h_s)}
	\end{cases}
\end{align}

В системе уравнений \eqref{eq:yz_R_s_y_0_z_0} $y_0$ и $z_0$ это координаты
центра радиуса скругления у образца, а $R_s$ это радиус скругления. Изображение образца
с указанием начала координат и центром скругления показано на Рис. \ref{fig:sample_and_leash}.

\begin{figure}[h]
	\begin{center}
		\includesvg[width=0.3\textwidth]{image/sample_leash.svg}
		\caption {Расположение осей координат относительно образца с рычагом}
		\label{fig:sample_and_leash}
    \end{center}
\end{figure}

Для рычага прямоугольного сечения к которому приложена сила по оси $y$, как показано
на Рис. \ref{fig:rectangularLeash_force_Y-axis}, момент инерции сечения $J(z)$ примет
вид \eqref{eq:Jz_rect_y-axis}. Момент силы $M(z)$ и модуль Юнга $E(z)$ останутся
таким же как показано в формулах \eqref{eq:Mz_rect_x-axis} и \eqref{eq:Ez_rect_x-axis}.
Зависимость $y(z)$ рассчитывается из системы уравнений \eqref{eq:yz_R_s_y_0_z_0}.

\begin{align}
	\label{eq:Jz_rect_y-axis}
	&J(z) =
	\begin{cases}
		\dfrac{2}{3} \cdot b_s \cdot y^3(z) \text{,} &\text{если }\ 0 \leqslant z < L_s \\[3ex]
		\dfrac{b_p \cdot a_p^3}{12} \text{,} &\text{если }\ L_s \leqslant z \leqslant L
	\end{cases}
\end{align}

Рассмотрим образец с прикреплённым нему рычагом круглого сечения.
На Рис. \ref{fig:circularLeash} представлены два чертежа с рычагом круглого сечения и
приложенной к нему силой силой $\vec{P}$ по \mbox{оси $x$} и по \mbox{оси $y$}.

\newpage
\begin{figure}[h]
	\begin{center}
		\hspace*{\fill}
		\begin{subfigure}{0.4\textwidth}
			\includesvg[width=\textwidth]{image/sampleRadiusWorkingPart_circularLeash_force_X-axis.svg}
			\caption {Сила приложена по оси $x$}
			\label{fig:circularLeash_force_X-axis}
		\end{subfigure}
		\hfill
		\begin{subfigure}{0.4\textwidth}
			\includesvg[width=\textwidth]{image/sampleRadiusWorkingPart_circularLeash_force_Y-axis.svg}
			\caption {Сила приложена по оси $y$}
			\label{fig:circularLeash_force_Y-axis}
		\end{subfigure}
		\hspace*{\fill}
		\caption {Образец с рычагом круглого сечения}
		\label{fig:circularLeash}
	\end{center}
\end{figure}

Для образца с прикреплённым к нему рычагом круглого сечения и силой приложенной
по оси $x$, как показано на Рис. \ref{fig:circularLeash_force_X-axis}, значения
переменных $M(z)$ и $E(z)$ рассчитваются по формулам \eqref{eq:Mz_rect_x-axis} и
\eqref{eq:Ez_rect_x-axis}. Момент инерции сечения $J(z)$ рассчитывается по
следующей формуле

\begin{align}
	\label{eq:Jz_circular_x-axis}
	&J(z) =
	\begin{cases}
		\dfrac{2}{3} \cdot b_s^3 \cdot y(z) \text{,} &\text{если }\ 0 \leqslant z < L_s \\[3ex]
		\dfrac{\pi}{64} \cdot D_p^4  \text{,} &\text{если }\ L_s \leqslant z \leqslant L
	\end{cases}
\end{align}

Для образца с прикреплённым к нему рычагом круглого сечения и силой приложенной
по оси $y$, как показано на Рис. \ref{fig:circularLeash_force_Y-axis}, значения
переменных $M(z)$ и $E(z)$ рассчитваются по формулам \eqref{eq:Mz_rect_x-axis} и
\eqref{eq:Ez_rect_x-axis}. Момент инерции сечения $J(z)$ определяется по
следующей формуле

\begin{align}
	\label{eq:Jz_circular_y-axis}
	&J(z) =
	\begin{cases}
		\dfrac{2}{3} \cdot b_s \cdot y^3(z) \text{,} &\text{если }\ 0 \leqslant z < L_s \\[3ex]
		\dfrac{\pi}{64} \cdot D_p^4  \text{,} &\text{если }\ L_s \leqslant z \leqslant L
	\end{cases}
\end{align}

В формулах расчёта момента инерции сечения \eqref{eq:Jz_circular_x-axis} и \eqref{eq:Jz_circular_y-axis}
зависимость $y(z)$ определяется из системы уравнений \eqref{eq:yz_R_s_y_0_z_0}.

\newpage
\subsection{Реализации расчёта в программе}

Поскольку в программе расчёт прогига $\nu$ по формуле \eqref{eq:iint_deflection} производится
численным методом (методом прямоугольников) то для расчёта шаг интегрирования $dz = \Delta z$
выбирается как сумма длин образца и рычага делёных на число $S_{int}$, которое
в программе принимает значения от $10^2$ до $10^8$. Формула для расчёта шага интегрирования
имеет следующий вид

\begin{align}
	\label{eq:integration_step}
	\Delta z &= \frac{L_s + L_p}{S_{int}}
\end{align}

Значение числа $S_{int}$ определяет точность с которой будет рассчитан прогиб и длительность самого расчёта.
Оптимальным значением этого числа, которое по умолчанию задано в программе, является $10^6$.
Если $S_{int}$ будет большо то время расчёта значительно увеличится, а если сделать меньше
то точность расчёта упадёт. Если в программе поставить галочку на против пункта
$"$Шаг интегрирования$"$ то можно задать шаг интегрирования в ручную.

Поскольку изначально значение силы $P$ неизвестно, а задано только механическое напряжение $\sigma$
в программе модуль силы $P$ берётся равным единице. В процессе расчёта
определяется величина прогиба $\nu_\tau$ при приложение к рычагу единичной силе.
Для расчёта $\nu_\tau$ в формуле \eqref{eq:iint_deflection}
заменим интегрирование на суммирование с шагом $\Delta z$.
Итоговая формула для расчёта прогиба $\nu_\tau$ примет следующий вид

\begin{align}
	\label{eq:sum_deflection_tau}
	\nu_\tau &=  \left| \sum_{i=0}^{S_{int}} \frac{M(z)}{E(z) \cdot J(z)} \cdot \Delta z \cdot \Delta z \right|
\end{align}

При расчёте $\nu_\tau$ по формуле \eqref{eq:sum_deflection_tau} производится определение
максимального механического напряжения $\sigma_{max}$ (\cite{2_} $\S$  65) в рабочей области образца, в цыкле
самого расчёта, по формуле

\begin{align}
	\label{eq:sigma_max}
	\sigma_{max} &=
	\begin{cases}
		\left| \dfrac{M(z)}{J(z)} \cdot \dfrac{b_s}{2} \right| \text{,} &\text{ если сила направлена по оси x} \\[3ex]
		\left| \dfrac{M(z)}{J(z)} \cdot y(z) \right| \text{,} &\text{ если сила направлена по оси y}
	\end{cases}
\end{align}

После расчёта прогиба $\nu_\tau$ при единичной силе и нахождения максимального механического
напряжения в рабочей части образца $\sigma_{max}$ \eqref{eq:sigma_max}, производится расчёт прогиба всей системы
$"$образец + рычаг$"$ при заданном механическом напряжении $\sigma$ по формуле

\begin{align}
	\label{eq:nu_stress}
	\nu &= \nu_\tau \cdot \frac{\sigma}{\sigma_{max}}
\end{align}

Если в программе поставить галочку на против пункта $"$Прогиб при силе один Ньютон$"$ то
программа после окончания расчёта в поле вывода дополнительно выдаст значения максимального механического
напряжения на образце $\sigma_{max}$, прогиба $\nu_\tau$ и $z_{max}$ при приложении единичной силы.
Величина $z_{max}$ определяет расстояние по оси $z$ где механическое напряжение максимально
то есть равно $\sigma_{max}$.

\newpage
Для проверки правильности расчёта прогиба $\nu$ при заданном механическом напряжении $\sigma$
в программе предусмотрен расчёт силы на конце рычага по формуле

\begin{align}
	\label{eq:force_sigma}
	P' &= P \cdot \frac{\sigma}{\sigma_{max}}
\end{align}

Величину силы $P'$ необходимо сравнивать с измеренной силой, которая получается
при прогибе образца на величину $\nu$. Если сила при величине прогиба $\nu$ не будет
соответствовать расчётной с точностью не хуже 3$\%$ то нужно перепроверить размеры, модуль Юнга у
образца и рычага или проверить образец на деффекты (трещины, вмятины и т.п.).

Если в программе поставить галочку у пункта $"$Параметры калибровки$"$ то появится
два поля ввода в которые можно ввести коэффициенты $k$ и $b$, полученные при калибровке конкретного
образца. Программа в поле вывода после расчёта выдаст чувствительность всей
измерительной системы, которая равна коэффициенту $k$. Данная опция нужна для
дальнейшего развития программы.












