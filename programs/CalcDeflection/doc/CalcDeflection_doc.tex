\documentclass[a4paper,12pt]{article}
\usepackage[utf8x]{inputenc}
\usepackage[english,russian]{babel}
\usepackage{indentfirst} % табуляция в начале абзаца
\usepackage{graphicx} % вставка изображений
\usepackage{wrapfig} % вставки изображений окруженных текстом
\usepackage{amsmath}
\usepackage{xcolor} % цвет для макросов
\usepackage{hyperref} % макросы
\usepackage{hyphenat} % макрос для сокращения слов
\usepackage[left=2cm, top=3cm, right=2cm, bottom=2cm]{geometry} % отступы по краям листа

\definecolor{linkcolor}{HTML}{009900} % цвет ссылок (зелёный)
\definecolor{urlcolor}{HTML}{009900} % цвет гиперссылок
%\DeclareUnicodeCharacter{00A0}{} % неразрывные пробелы в кодировке UTF-8
\DeclareUnicodeCharacter{200E}{} % написание с лева на право
\hypersetup{pdfstartview=FitH, linkcolor=linkcolor, urlcolor=urlcolor, colorlinks=true} % настройка цветов для выделения ссылок

% титульный лист
\title{Описание алгоритмов расчёта программы CalcDeflection 0.2}
\author{Ветров Д.Н.}
\date{\today}

\begin{document}

	% Убираем нумерацию с титульного листа
	\begin{titlepage}
		\maketitle % титульный листа
		\thispagestyle{empty} % убираем нумерацию титульного листа
	\end{titlepage}
	\pagenumbering{arabic} % нумерация арабскими цыфрами
	\newpage

	% содержание
	\tableofcontents

	% о программе
	%\include{aboutProg_ru}

	% описание расчёта по формуле Уилера
	%\include{coilWheeler_ru}

	% расчёт по формуле для замкнутых витков
	%\include{coilClosTurns_ru}

	% список литературы
	%\include{references_ru}
\end{document}

%\section{Раздел}
%\label{ex:section}
%\subsection{Подраздел}
%\label{ex:subsection}
%\subsubsection[<<Подподраздел>>]{Что-то более мелкое чем подраздел}
%\label{ex:subsubsection}
%\paragraph{Параграф}
%\label{ex:paragraph}
%\subparagraph{Подпараграф}
%\label{ex:subparagraph}

